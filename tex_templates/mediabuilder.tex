% mediabuilder.tex is an external preamble that can be included in 
% pandoc templates to improve the way tex templates are rendered
% This file contains a bunch of custom includes that I use across many tex
% templates

% Stuff added for pandoc

% ALGORITHMS -------------------------------------------------------------------

% These two packages are used to write algorithms
\usepackage{algorithm}
\usepackage{algpseudocode}


% HYPERLINKS -------------------------------------------------------------------

% Setting up hyperref so that links can be populated
\usepackage{ifxetex,ifluatex}
\ifxetex
  \usepackage[setpagesize=false, % page size defined by xetex
              unicode=false, % unicode breaks when used with xetex
              xetex]{hyperref}
\else
  \usepackage[unicode=true]{hyperref}
\fi
\hypersetup{breaklinks=true, pdfborder={0 0 0}}


% FIGURES ----------------------------------------------------------------------
% used for wrapfig
\usepackage{wrapfig}

\usepackage{dblfloatfix}

% To handle multicolumn figures
% This lets you do \Begin{widefig}
\let\Begin\begin
\let\End\end
\newenvironment{widefig}{
	\renewenvironment{figure}{\begin{figure*}\centering}{\end{figure*}}
	\renewenvironment{table}{\begin{table*}\centering}{\end{table*}}
}

\newenvironment{widefigbot}{
  \renewenvironment{figure}{\begin{figure*}[!b]\centering}{\end{figure*}}
  \renewenvironment{table}{\begin{table*}\centering}{\end{table*}}
}

% Scale images if necessary, so that they will not overflow the page
% margins by default, and it is still possible to overwrite the defaults
% using explicit options in \includegraphics[width, height, ...]{}
\usepackage{graphicx}
\makeatletter
\def\maxwidth{\ifdim\Gin@nat@width>\linewidth\linewidth\else\Gin@nat@width\fi}
\def\maxheight{\ifdim\Gin@nat@height>\textheight\textheight\else\Gin@nat@height\fi}
\makeatother
\setkeys{Gin}{width=\maxwidth,height=\maxheight,keepaspectratio}

% Old way to do this:
% $if(graphics)$
% \usepackage{graphicx}
% % We will generate all images so they have a width \maxwidth. This means
% % that they will get their normal width if they fit onto the page, but
% % are scaled down if they would overflow the margins.
% \makeatletter
% \def\maxwidth{\ifdim\Gin@nat@width>\linewidth\linewidth
% \else\Gin@nat@width\fi}
% \makeatother
% \let\Oldincludegraphics\includegraphics
% \renewcommand{\includegraphics}[1]{\Oldincludegraphics[width=\maxwidth]{#1}}
% $endif$


% When I was using the multicol environment, I had to redefine the figure
% environment so there's no float (what pandoc does by default)
% to make it compatible with the multicol package. This is how I did that.
% This led to several problems, such as figures showing up out-of-order. In the
% end, because of this, I abandoned the multicol package and switched back to
% the basic, builtin twocol environment. Leaving this here in case I ever need
% multicol again:
% \usepackage{float}
% \let\origfigure\figure
% \let\endorigfigure\endfigure
% \renewenvironment{figure}[1][2] {
%     \expandafter\origfigure\expandafter[H]
% } {
%     \endorigfigure
% }

% % Also for the table
% \let\origtable\table
% \let\endorigtable\endtable
% \renewenvironment{table}[1][2] {
%     \expandafter\origtable\expandafter[H]
% } {
%     \endorigtable
% }





% LISTS ----------------------------------------------------------------------

\let\tightlist\relax

% \providecommand{\tightlist}{
%   \setlength{\itemsep}{0pt}
%   \setlength{\parskip}{0pt}}

\providecommand{\tightlist}{
  \setlength{\topsep}{1em}
  \setlength{\partopsep}{1em}
  \setlength{\parskip}{1em}
  \setlength{\parsep}{1em}
  \setlength{\parindent}{0pt}
  \setlength{\itemsep}{0pt}
  \setlength{\itemindent}{0pt}
  \setlength{\leftmargin}{0pt}
}

% make bullets smaller
\def\labelitemi{\raisebox{0.5ex}{\tiny\textbullet}}
% this itemindent code adjusts the indentation of bullet lists.

% Redefine labelwidth for lists; otherwise, the enumerate package will cause
% markers to extend beyond the left margin.
\makeatletter\AtBeginDocument{%
  \renewcommand{\@listi}
    {\setlength{\labelwidth}{4em}}
}\makeatother
\usepackage{enumerate}



% TABLES -----------------------------------------------------------------------
% longtable is required for markdown-format tables to be rendered in latex
\usepackage{longtable}

% LAYOUT -----------------------------------------------------------------------
\def\topfraction{0.9} % 90 percent of the page may be used by floats on top
\def\bottomfraction{0.9} % 90 percent of the page may be used by floats at the bottom

\clubpenalty=9996
\widowpenalty=9999
\brokenpenalty=4991
\predisplaypenalty=10000
\postdisplaypenalty=1549
\displaywidowpenalty=1602

\setlength{\parindent}{0pt}
\setlength{\parskip}{6pt plus 2pt minus 1pt}
\setlength{\emergencystretch}{3em}  % prevent overfull lines


% FONTS ---

\renewcommand{\rmdefault}{phv} % Arial
\renewcommand{\sfdefault}{phv} % Arial
% \renewcommand{\rmdefault}{bch} % Charter
% \renewcommand{\sfdefault}{bch} % Charter


% CSV INCLUDES ---

% for including csv files
\usepackage{tikz}  % required to fix incompatibility with ctable: https://github.com/T-F-S/csvsimple/issues/9
\usepackage{csvsimple}

% for toprule and midrule
\usepackage{booktabs}

% formatting csv tables with 'catch-all' variable column widths
\usepackage{tabularx}
